% -*- coding: utf-8; -*-

\chapter{Situação Atual}

\label{cha:Situação Atual}

A inteligência artificial na medicina consiste no uso de algoritmos de aprendizado de máquina, aliado com grandes bases de dados, para promover análises mais precisas de exames e melhorar a experiência dos pacientes. Muitos desses modelos de aprendizado fornecem suporte para várias partes do mundo, tanto dentro de hospitais ajuando médicos a tomarem melhores decisões, como também em pesquisas de novos fármacos.

Primeiramente, podemos perceber uma frente bem relevante na área de desenvolvimento de novos medicamentos. O processo de criação até teste que é realizado hoje é extremamente custoso, podendo levar anos de pesquisa. Segundo o artigo feito por Mathew Chun \cite{HARV1}, com a chegada de novos algorítmos de redes neurais, como NLP, diversas áreas desse ramo foram afetas e estão contribuindo positivamente para o aperfeiçoamento de novos remédios. Melhoras significativas foram notadas na aquisição de melhores alvos de genes para servirem de contrariantes para determinas doença; Predição de propriedades dos fármacos antes mesmo da testagem; Geração de padrões moleculares de remédios, nunca antes vistos.

O uso da IA para detecção de doenças e produção de diagnósticos mais precisos se tornou um alido para diferentes clínicos, contribuindo para o aumento de sua produtividade. Justamente pelo fato de máquinas não terem as mesmas limitações fisícas que um humano normal, uma IA pode coletar dados de pacientes como, batimentos cardíacos, respirações por minuto, informações sobre a pressão, enquanto o médico não estiver presente e notificar quando notar algo fora dos padrões. Como dito no estudo da IBM, um cliente da empresa conseguiu desenvolver um modelo preditivo de inteligência artificial para detectar condições de sepse grave em bebês pré-maturos. \cite{IBM1} %aumentar isso%

A inteligência artificial também pode ser aplicada no exame de imagens médicas. Esse ramo é muito importante na medicina, já que elas são necessárias para a visualização de órgãos internos com o objetivo de detectar anomalias em sua estrutura ou funcionamento. Muitos dispositivos podem ser utilizados, gerando imagens ou vídeos de qualidade, sendo eles, raios-x, tomografias computadorizadas, ressonâncias magnéticas, tomografias por emissão de pósitrons (PET), e ultrassonografias. Após a captura, essas imagens devem ser analisadas e compreendidas para a detecção precisa de anormalidades. Caso seja identificada alguma inconformidade, devem ser apresentadas sua localização exata, tamanho e forma da anomalia. Hoje, sistemas de saúdes mais inteligentes, tem como objetivo aplicar IA para realizar essas tarefas de análise, que antes eram feitas por médicos treinados com sua experiência adquirida ao longo dos anos.


Estudos indicam que o uso de IA's em imagens médicas podem ser tão eficiente quanto um radiologista detectando sinais de câncer de mama em um exame. De acordo com o centro de tecnologia de Massachusetts, foi criado um modelo de aprendizado de máquina capaz de detectar o crescimento de um tumor maligno na mama de uma paciente, 5 anos antes de se formar completamente e deixá-la em condições graves. \cite{MIT1} O modelo foi treinado com dados de mais de 60.000 pacientes e, o mesmo conseguiu detectar padrões sutis no tecido mamário que indicariam futuros casos de tumores na região. 


%https://www.ibm.com/topics/artificial-intelligence-medicine#:~:text=How%20is%20artificial%20intelligence%20used,health%20outcomes%20and%20patient%20experiences.

%%https://www.sbmastologia.com.br/inteligencia-artificial-preve-cancer-de-mama-cinco-anos-antes/